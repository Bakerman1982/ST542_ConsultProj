% Options for packages loaded elsewhere
\PassOptionsToPackage{unicode}{hyperref}
\PassOptionsToPackage{hyphens}{url}
\PassOptionsToPackage{dvipsnames,svgnames,x11names}{xcolor}
%
\documentclass[
  11pt,
  letterpaper,
  DIV=11,
  numbers=noendperiod]{scrartcl}

\usepackage{amsmath,amssymb}
\usepackage{iftex}
\ifPDFTeX
  \usepackage[T1]{fontenc}
  \usepackage[utf8]{inputenc}
  \usepackage{textcomp} % provide euro and other symbols
\else % if luatex or xetex
  \usepackage{unicode-math}
  \defaultfontfeatures{Scale=MatchLowercase}
  \defaultfontfeatures[\rmfamily]{Ligatures=TeX,Scale=1}
\fi
\usepackage{lmodern}
\ifPDFTeX\else  
    % xetex/luatex font selection
    \setmainfont[]{Times New Roman}
\fi
% Use upquote if available, for straight quotes in verbatim environments
\IfFileExists{upquote.sty}{\usepackage{upquote}}{}
\IfFileExists{microtype.sty}{% use microtype if available
  \usepackage[]{microtype}
  \UseMicrotypeSet[protrusion]{basicmath} % disable protrusion for tt fonts
}{}
\makeatletter
\@ifundefined{KOMAClassName}{% if non-KOMA class
  \IfFileExists{parskip.sty}{%
    \usepackage{parskip}
  }{% else
    \setlength{\parindent}{0pt}
    \setlength{\parskip}{6pt plus 2pt minus 1pt}}
}{% if KOMA class
  \KOMAoptions{parskip=half}}
\makeatother
\usepackage{xcolor}
\usepackage[margin=1in]{geometry}
\setlength{\emergencystretch}{3em} % prevent overfull lines
\setcounter{secnumdepth}{5}
% Make \paragraph and \subparagraph free-standing
\makeatletter
\ifx\paragraph\undefined\else
  \let\oldparagraph\paragraph
  \renewcommand{\paragraph}{
    \@ifstar
      \xxxParagraphStar
      \xxxParagraphNoStar
  }
  \newcommand{\xxxParagraphStar}[1]{\oldparagraph*{#1}\mbox{}}
  \newcommand{\xxxParagraphNoStar}[1]{\oldparagraph{#1}\mbox{}}
\fi
\ifx\subparagraph\undefined\else
  \let\oldsubparagraph\subparagraph
  \renewcommand{\subparagraph}{
    \@ifstar
      \xxxSubParagraphStar
      \xxxSubParagraphNoStar
  }
  \newcommand{\xxxSubParagraphStar}[1]{\oldsubparagraph*{#1}\mbox{}}
  \newcommand{\xxxSubParagraphNoStar}[1]{\oldsubparagraph{#1}\mbox{}}
\fi
\makeatother

\usepackage{color}
\usepackage{fancyvrb}
\newcommand{\VerbBar}{|}
\newcommand{\VERB}{\Verb[commandchars=\\\{\}]}
\DefineVerbatimEnvironment{Highlighting}{Verbatim}{commandchars=\\\{\}}
% Add ',fontsize=\small' for more characters per line
\usepackage{framed}
\definecolor{shadecolor}{RGB}{241,243,245}
\newenvironment{Shaded}{\begin{snugshade}}{\end{snugshade}}
\newcommand{\AlertTok}[1]{\textcolor[rgb]{0.68,0.00,0.00}{#1}}
\newcommand{\AnnotationTok}[1]{\textcolor[rgb]{0.37,0.37,0.37}{#1}}
\newcommand{\AttributeTok}[1]{\textcolor[rgb]{0.40,0.45,0.13}{#1}}
\newcommand{\BaseNTok}[1]{\textcolor[rgb]{0.68,0.00,0.00}{#1}}
\newcommand{\BuiltInTok}[1]{\textcolor[rgb]{0.00,0.23,0.31}{#1}}
\newcommand{\CharTok}[1]{\textcolor[rgb]{0.13,0.47,0.30}{#1}}
\newcommand{\CommentTok}[1]{\textcolor[rgb]{0.37,0.37,0.37}{#1}}
\newcommand{\CommentVarTok}[1]{\textcolor[rgb]{0.37,0.37,0.37}{\textit{#1}}}
\newcommand{\ConstantTok}[1]{\textcolor[rgb]{0.56,0.35,0.01}{#1}}
\newcommand{\ControlFlowTok}[1]{\textcolor[rgb]{0.00,0.23,0.31}{\textbf{#1}}}
\newcommand{\DataTypeTok}[1]{\textcolor[rgb]{0.68,0.00,0.00}{#1}}
\newcommand{\DecValTok}[1]{\textcolor[rgb]{0.68,0.00,0.00}{#1}}
\newcommand{\DocumentationTok}[1]{\textcolor[rgb]{0.37,0.37,0.37}{\textit{#1}}}
\newcommand{\ErrorTok}[1]{\textcolor[rgb]{0.68,0.00,0.00}{#1}}
\newcommand{\ExtensionTok}[1]{\textcolor[rgb]{0.00,0.23,0.31}{#1}}
\newcommand{\FloatTok}[1]{\textcolor[rgb]{0.68,0.00,0.00}{#1}}
\newcommand{\FunctionTok}[1]{\textcolor[rgb]{0.28,0.35,0.67}{#1}}
\newcommand{\ImportTok}[1]{\textcolor[rgb]{0.00,0.46,0.62}{#1}}
\newcommand{\InformationTok}[1]{\textcolor[rgb]{0.37,0.37,0.37}{#1}}
\newcommand{\KeywordTok}[1]{\textcolor[rgb]{0.00,0.23,0.31}{\textbf{#1}}}
\newcommand{\NormalTok}[1]{\textcolor[rgb]{0.00,0.23,0.31}{#1}}
\newcommand{\OperatorTok}[1]{\textcolor[rgb]{0.37,0.37,0.37}{#1}}
\newcommand{\OtherTok}[1]{\textcolor[rgb]{0.00,0.23,0.31}{#1}}
\newcommand{\PreprocessorTok}[1]{\textcolor[rgb]{0.68,0.00,0.00}{#1}}
\newcommand{\RegionMarkerTok}[1]{\textcolor[rgb]{0.00,0.23,0.31}{#1}}
\newcommand{\SpecialCharTok}[1]{\textcolor[rgb]{0.37,0.37,0.37}{#1}}
\newcommand{\SpecialStringTok}[1]{\textcolor[rgb]{0.13,0.47,0.30}{#1}}
\newcommand{\StringTok}[1]{\textcolor[rgb]{0.13,0.47,0.30}{#1}}
\newcommand{\VariableTok}[1]{\textcolor[rgb]{0.07,0.07,0.07}{#1}}
\newcommand{\VerbatimStringTok}[1]{\textcolor[rgb]{0.13,0.47,0.30}{#1}}
\newcommand{\WarningTok}[1]{\textcolor[rgb]{0.37,0.37,0.37}{\textit{#1}}}

\providecommand{\tightlist}{%
  \setlength{\itemsep}{0pt}\setlength{\parskip}{0pt}}\usepackage{longtable,booktabs,array}
\usepackage{calc} % for calculating minipage widths
% Correct order of tables after \paragraph or \subparagraph
\usepackage{etoolbox}
\makeatletter
\patchcmd\longtable{\par}{\if@noskipsec\mbox{}\fi\par}{}{}
\makeatother
% Allow footnotes in longtable head/foot
\IfFileExists{footnotehyper.sty}{\usepackage{footnotehyper}}{\usepackage{footnote}}
\makesavenoteenv{longtable}
\usepackage{graphicx}
\makeatletter
\newsavebox\pandoc@box
\newcommand*\pandocbounded[1]{% scales image to fit in text height/width
  \sbox\pandoc@box{#1}%
  \Gscale@div\@tempa{\textheight}{\dimexpr\ht\pandoc@box+\dp\pandoc@box\relax}%
  \Gscale@div\@tempb{\linewidth}{\wd\pandoc@box}%
  \ifdim\@tempb\p@<\@tempa\p@\let\@tempa\@tempb\fi% select the smaller of both
  \ifdim\@tempa\p@<\p@\scalebox{\@tempa}{\usebox\pandoc@box}%
  \else\usebox{\pandoc@box}%
  \fi%
}
% Set default figure placement to htbp
\def\fps@figure{htbp}
\makeatother

\KOMAoption{captions}{tableheading,figureheading}
\makeatletter
\@ifpackageloaded{caption}{}{\usepackage{caption}}
\AtBeginDocument{%
\ifdefined\contentsname
  \renewcommand*\contentsname{Table of contents}
\else
  \newcommand\contentsname{Table of contents}
\fi
\ifdefined\listfigurename
  \renewcommand*\listfigurename{List of Figures}
\else
  \newcommand\listfigurename{List of Figures}
\fi
\ifdefined\listtablename
  \renewcommand*\listtablename{List of Tables}
\else
  \newcommand\listtablename{List of Tables}
\fi
\ifdefined\figurename
  \renewcommand*\figurename{Figure}
\else
  \newcommand\figurename{Figure}
\fi
\ifdefined\tablename
  \renewcommand*\tablename{Table}
\else
  \newcommand\tablename{Table}
\fi
}
\@ifpackageloaded{float}{}{\usepackage{float}}
\floatstyle{ruled}
\@ifundefined{c@chapter}{\newfloat{codelisting}{h}{lop}}{\newfloat{codelisting}{h}{lop}[chapter]}
\floatname{codelisting}{Listing}
\newcommand*\listoflistings{\listof{codelisting}{List of Listings}}
\makeatother
\makeatletter
\makeatother
\makeatletter
\@ifpackageloaded{caption}{}{\usepackage{caption}}
\@ifpackageloaded{subcaption}{}{\usepackage{subcaption}}
\makeatother

\usepackage{bookmark}

\IfFileExists{xurl.sty}{\usepackage{xurl}}{} % add URL line breaks if available
\urlstyle{same} % disable monospaced font for URLs
\hypersetup{
  pdftitle={Exploratory Data Analysis},
  pdfauthor={Brock Akerman; Hanan Ali; Taylor Cesarski},
  colorlinks=true,
  linkcolor={blue},
  filecolor={Maroon},
  citecolor={Blue},
  urlcolor={Blue},
  pdfcreator={LaTeX via pandoc}}


\title{Exploratory Data Analysis}
\usepackage{etoolbox}
\makeatletter
\providecommand{\subtitle}[1]{% add subtitle to \maketitle
  \apptocmd{\@title}{\par {\large #1 \par}}{}{}
}
\makeatother
\subtitle{Educators and Employers Survey Data Exploration}
\author{Brock Akerman \and Hanan Ali \and Taylor Cesarski}
\date{2025-06-15}

\begin{document}
\maketitle

\renewcommand*\contentsname{Table of contents}
{
\hypersetup{linkcolor=}
\setcounter{tocdepth}{2}
\tableofcontents
}

\newpage

\section{Abstract}\label{abstract}

\section{Introduction}\label{introduction}

Our researcher, Dr.~Ross-Estrada has distributed a survey to two
distinct participant groups:

\begin{itemize}
\item
  Educators who teach in the field of dental veterinary medicine (DVM),
  and
\item
  Employers who have recently hired graduates from DVM programs.
\end{itemize}

These two respondent groups provide us with two separate
datasets---educators and employers---each with its own structure and
variables. While there is some overlap between them, differences in
content and context mean that we will treat these datasets separately in
most of our analysis. Dr.~Ross-Estrada wishes to extract insights about
the two groups and their perspectives concerning training and
capabilities of new graduates of dental veterinarian medicine programs.
The survey was conducted during the summer of 2024 using Qualtrics--an
experience management software service. Selection of the participants
was not conducted randomly; instead our researchers network was used.

\section{Initial Data Inspection}\label{initial-data-inspection}

The best approach is to first get a large look from the top down on
these two datasets to see what we are working with. Let us first take a
look at the Education dataset.

\begin{Shaded}
\begin{Highlighting}[]
\NormalTok{Educator\_Data }\OtherTok{\textless{}{-}} \FunctionTok{read\_csv}\NormalTok{(}\StringTok{"PCVE\_Dentistry\_Survey.csv"}\NormalTok{, }\AttributeTok{show\_col\_types =} \ConstantTok{FALSE}\NormalTok{)}
\end{Highlighting}
\end{Shaded}

\begin{verbatim}
New names:
* `Q46` -> `Q46...18`
* `Q45` -> `Q45...37`
* `Q46` -> `Q46...38`
* `Q45` -> `Q45...159`
\end{verbatim}

\begin{Shaded}
\begin{Highlighting}[]
\FunctionTok{dim}\NormalTok{(Educator\_Data)}
\end{Highlighting}
\end{Shaded}

\begin{verbatim}
[1]  45 171
\end{verbatim}

\begin{Shaded}
\begin{Highlighting}[]
\FunctionTok{head}\NormalTok{(Educator\_Data)}
\end{Highlighting}
\end{Shaded}

\begin{verbatim}
# A tibble: 6 x 171
  StartDate    EndDate Status IPAddress Progress Duration (in seconds~1 Finished
  <chr>        <chr>   <chr>  <chr>     <chr>    <chr>                  <chr>   
1 "Start Date" "End D~ "Resp~ "IP Addr~ "Progre~ "Duration (in seconds~ "Finish~
2 "{\"ImportI~ "{\"Im~ "{\"I~ "{\"Impo~ "{\"Imp~ "{\"ImportId\":\"dura~ "{\"Imp~
3 "8/8/2024 1~ "8/8/2~ "0"    "128.84.~ "100"    "600"                  "1"     
4 "8/8/2024 1~ "8/8/2~ "0"    "192.0.2~ "100"    "29"                   "1"     
5 "8/8/2024 1~ "8/8/2~ "0"    "38.175.~ "100"    "801"                  "1"     
6 "8/8/2024 1~ "8/8/2~ "0"    "152.1.8~ "100"    "3375"                 "1"     
# i abbreviated name: 1: `Duration (in seconds)`
# i 164 more variables: RecordedDate <chr>, ResponseId <chr>,
#   RecipientLastName <chr>, RecipientFirstName <chr>, RecipientEmail <chr>,
#   ExternalReference <chr>, LocationLatitude <chr>, LocationLongitude <chr>,
#   DistributionChannel <chr>, UserLanguage <chr>, Q46...18 <chr>, Q44 <chr>,
#   Q3 <chr>, Q3_13_TEXT <chr>, Q4_1 <chr>, Q4_2 <chr>, Q4_3 <chr>, Q4_4 <chr>,
#   Q4_5 <chr>, Q4_6 <chr>, Q4_7 <chr>, Q4_8 <chr>, Q4_9 <chr>, ...
\end{verbatim}

The results show a table with 45 rows and 171 columns. Retrieving a
sample of this table reveals that the top two rows are subheaders likely
included as part of the Qualtric survey output and not data used to
measure the sentiment about dental veterinarian students knowledge.

We will do the same for the Employers dataset. -Load the data and
inspect dimensions (n rows × p columns) -Check variable names and types
(numeric, factor, character, date, etc.) -Print first few rows (head())
and summary statistics -Identify response (outcome) and predictor
(explanatory) variables

\begin{Shaded}
\begin{Highlighting}[]
\NormalTok{Employer\_Data }\OtherTok{\textless{}{-}} \FunctionTok{read\_csv}\NormalTok{(}\StringTok{"Employer\_Dentistry\_Survey.csv"}\NormalTok{, }\AttributeTok{show\_col\_types =} \ConstantTok{FALSE}\NormalTok{)}
\FunctionTok{dim}\NormalTok{(Employer\_Data)}
\end{Highlighting}
\end{Shaded}

\begin{verbatim}
[1]  31 176
\end{verbatim}

\section{Missing Data}\label{missing-data}

Count missing values per column and per row Visualize missingness
patterns (e.g., heatmap or naniar/VIM plots) Check for systematic
missingness (by group, time, etc.) Decide: drop, impute, flag, or model
missingness?

\section{Univariate Analysis}\label{univariate-analysis}

\begin{itemize}
\tightlist
\item
  For each variable: Compute summary stats: mean, median, SD, range, IQR
  Plot distribution: histograms (numeric), bar plots (categorical)
  Identify outliers or skewness Check for unusual values or coding
  errors
\end{itemize}

\section{Bivariate Analysis}\label{bivariate-analysis}

\begin{itemize}
\tightlist
\item
  Numeric vs Numeric Scatterplots with smoothing (e.g., LOESS)
  Correlation coefficients (Pearson/Spearman)
\item
  Categorical vs Numeric Boxplots or violin plots Group means + CI/error
  bars ANOVA or t-tests (exploratory, not confirmatory)
\item
  Categorical vs Categorical Cross-tabulations Chi-square or Fisher's
  tests (exploratory)
\end{itemize}

\section{Multivariate Structure}\label{multivariate-structure}

Correlation heatmap (numeric variables) Principal Component Analysis
(PCA) or t-SNE (if high-dimensional) Pair plots / scatterplot matrix
Check multicollinearity (VIFs, condition index)

\subsection{Time/Spatial Data (if
needed)}\label{timespatial-data-if-needed}

Time series plots Trends, seasonality, anomalies Autocorrelation, lag
plots Maps or geospatial distribution

\section{Data Integrity Checks}\label{data-integrity-checks}

Check for duplicates (rows, IDs) Validate ranges against expected values
Consistency across related variables (e.g., start\_date \textless{}
end\_date) Confirm units and scales are consistent

\section{Documentation \& client
Communication}\label{documentation-client-communication}

Create a clean report with summary tables and visualizations Highlight
any data issues that could affect modeling Document assumptions,
decisions (e.g., handling of missing or outliers) Make notes on
variables of interest for modeling




\end{document}
